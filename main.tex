\documentclass[12px]{article}
\usepackage[utf8x]{inputenc}
\usepackage[portuges]{babel}
\usepackage[svgnames]{xcolor}
\usepackage{graphicx}
\usepackage{a4wide}
\usepackage{float}
\usepackage[export]{adjustbox}
\usepackage{spverbatim}

\usepackage{enumitem}
\setdescription{leftmargin=\parindent,labelindent=\parindent}

\newcommand*{\plogo}{\fbox{$\mathcal{PL}$}}
\newcommand*{\titleTH}{\begingroup
\raggedleft
\vspace*{\baselineskip}

{\Large Universidade do Minho}\\[0.167\textheight]
{\LARGE\bfseries Projeto Integrado}\\[\baselineskip]
{\Huge Gestão de Trabalhos Práticos}\\[\baselineskip]
{\Large \textit{Engenharia de Linguagens 13/14}}\par

\vfill

\Large{André Santos pg25329\\Daniel Carvalho a61008\\Ricardo Branco pg25339}\par

\vspace*{3\baselineskip}
\endgroup}

\begin{document} 
\thispagestyle{empty}
\titleTH
\newpage

\section{Resumo}

  Este relatório cobre todo o processo de planeamento, desenvolvimento e
  documentação de um sistema que resolva um problema abordado na UCE de
  Engenharia de Linguagens na forma de Projecto Integrado, nomeadamente
  a gestão de trabalhos práticos académicos desde a sua criação pelos docentes, à resolução pelos alunos
  e à publicação dos mesmos ao resto da comunidade \textit{online}.

  A abordagem na resolução deste projecto passa por aliar todos os conhecimentos
  adquiridos nos módulos integrantes de Engenharia de Linguagens, passando por
 Engenharia Gramatical, Scripting no Processamento de Linguagem Natural, Processamento
 Estruturado de Documentos e Análise e Transformação de Software.

 Neste sentido pretende-se aliar ao desenvolvimento deste sistema conhecimentos
 como gramáticas de atributos, ambientes de desenvolvimento estruturais e
 orientados à semântica, representação de manipulação de conhecimento com
 eficiência, automatização de tarefas e transformações, utilização de expressões
 regulares, linguagens DSL, corpora, automatização de testes para diferentes
 linguagens de programação, manipulação de documentos estruturados, utilização
 de XML, XSL e XSLFO, documentos anotados, publicação de conhecimento na web,
 entre outros.

\newpage

\tableofcontents
\newpage

\begin{abstract}
  Este relatório cobre todo o processo de planeamento, desenvolvimento e 
  documentação de um sistema que resolva um problema abordado na UCE de 
  Engenharia de Linguagens na forma de Projecto Integrador, nomeadamente
  a gestão de trabalhos práticos académicos desde a sua criação pelos docentes, à resolução pelos alunos
  e à publicação dos mesmos ao resto da comunidade \textit{online}.
  
  A abordagem na resolução deste projecto passa por aliar todos os conhecimentos 
  adquiridos nos módulos integrantes de Engenharia de Linguagens, passando por 
 Engenharia Gramatical, Scripting no Processamento de Linguagem Natural, Processamento
 Estruturado de Documentos e Análise e Transformação de Software.
 
 Neste sentido pretende-se aliar ao desenvolvimento deste sistema conhecimentos 
 como gramáticas de atributos, ambientes de desenvolvimento estruturais e 
 orientados à semântica, representação de manipulação de conhecimento com 
 eficiência, automatização de tarefas e transformações, utilização de expressões 
 regulares, linguagens DSL, corpora, automatização de testes para diferentes 
 linguagens de programação, manipulação de documentos estruturados, utilização 
 de XML, XSL e XSLFO, documentos anotados, publicação de conhecimento na web, 
 entre outros.
 
\end{abstract}

\section{Introdução}

Em termos gerais, este projeto baseia-se num sistema de informação que permita a gestão de trabalhos 
práticos de unidades curriculares de alunos do ensino superior.

A ideia central é permitir receber os trabalhos (programa e relatório) entregues por cada grupo e 
permitir associar a cada submissão os comentários e a classificação atribuída pelo docente, 
permitindo gerar a pauta da turma. No entanto de forma a garantir um maior 
espetro de usabilidade, olhar para o sistema de forma a poder suportar 
trabalhos práticos num formato mais genérico que permita não só receber 
trabalhos de programação mas todo o tipo de trabalhos práticos produzidos.
Não obstante, no que diz respeito aos trabalhos de programação, serão tidas em 
atenção algumas funcionalidades que melhorem o tratamento deste tipo específico 
de trabalhos fornecendo ferramentas para testes automáticos ao código submetido 
pelos alunos, que avaliam o funcionamento desejado ou não consoante as 
especificações do docente.

Para este sistema ser completo e funcional deve também garantir funcionalidades como
a gestão de unidades curriculares e a sua equipa docente,  gestão de turnos e dos alunos inscritos, e a
criação de trabalhos práticos dentro da unidade curricular.

No que diz respeito à implementação, este sistema é suportado por uma aplicação \textit{web}, acessível
nos vários tipos de dispositivos (computador, \textit{tablet}, \textit{smartphone}), deverá garantir a persistência de 
dados numa base de dados relacional, deverá garantir a interoperabilidade com outros sistemas para
importação ou exportação de dados, bem como permitir a publicação de pautas em vários formatos.

Noutra perspetiva é necessário tratar os trabalhos práticos disponibilizados na aplicação
e criar um sistema que sirva de repositório digital dos mesmos. Na sua implementação é necessário
garantir um \textit{backoffice} que permita aos professores fazer a gestão das unidades curriculares
de que são responsáveis, assim como dos turnos e projetos da mesma. É também requerido garantir
aos alunos a gestão dos seus grupos dentro dos projetos, assim como garantir a funcionalidade
de entrega de trabalhos práticos.
 
 Por fim, torna-se imperativo garantir a todos os utilizadores do sistema, a possibilidade de procura
 e consulta de todos os projetos disponíveis para o público.
 
 Posto isto, o sistema deve ter como a estrutura do modelo de referência internacional OAIS 
 (\textit{Open Archive Information System}).
 Nesta estrutura existem três organismos que fazem funcionar o sistema. Os produtores, que 
 alimentam o sistema com os trabalhos práticos, os administradores que fazem a gestão e administração
 dos trabalhos práticos e os consumidores que irão usufruir dos trabalhos práticos. 
 
No que concerne à validação dos projectos submetidos há a necessidade de verificações e 
criação de um modelo genérico a que todas as submissões terão de obedecer. Neste processo
é necessário tratar-se do SIP (\textit{Submission Information Package}) na ingestão do sistema dos
trabalhos práticos, que após tratamento da informação se transforma num AIP (\textit{Archival 
Information Package}) e então arquivado, e por fim a aplicação disponibiliza o trabalho no formato
de um DIP (\textit{Dissemination Information Package}) pronto a ser disseminado e consumido pelos
utilizadores do sistema.

\newpage
\section{Plano de trabalho}
Tendo em conta a dimensão da aplicação foi criado um plano de trabalho onde mapeámos as tarefas
de desenvolvimento no tempo de forma contínua tornado assim possível adaptar um ritmo de trabalho
consistente com os prazos de entrega de cada fase e então cumprir as metas do projecto.
Nesse sentido dividimos o projecto em três fases, sendo elas a análise de requisitos, o
desenvolvimento e a documentação.
No que diz respeito à análise de requitos, esta fase engloba todo o levantamento 
de funcionalidades assim como todo o planeamento e estruturação do projecto como
criação de modelos de domínio, diagramas de classes, modelação do repositório de dados,
análise da concorrência, plano tecnológico, etc.
Para esta fase de análise e modelação foi criado um plano de um mês onde todas as tarefas 
teriam de estar concluídas. Na imagem seguinte é possível ver um diagrama de gannt com
o planeamento inicial.

\section{Análise do problema}

No meio académico, principalmente em áreas mais focadas numa vertente prática, muitas vezes
os alunos são postos à prova através de trabalhos práticos. Neste caso, após a avaliação, os trabalhos
práticos são arquivados pelos docentes e esquecidos permanentemente numa prateleira, até um dia
serem deitados fora.

No sentido de promover a investigação e a partilha de conhecimento, torna-se não só necessário
criar uma plataforma de gestão de disciplinas universitárias e os seus trabalhos práticos, como também
tornar possível a partilha de todo o conhecimento que é inevitavelmente inerente aos resultados
do desenvolvimento destes.

No momento as ferramentas que existem no mercado, se por um lado permitem toda a gestão de
unidades curriculares de uma universidade, pecam por se focarem por um funcionamento fechado
para dentro da instituição e consequentemente fechado para dentro dos cursos e das unidades curriculares.
Nesse sentido o sistema retratado neste relatório, preenche a lacuna existente entre o que é desenvolvido
num âmbito académico e o resto da população.

Como resultado deste tipo de ponte entre os dois mundos, espera-se conseguir manter uma plataforma
que não só incentive a investigação tanto dentro do mundo académico, como profissional e individual,
como também incentive a partilha de conhecimento e fomente o desenvolvimento das
áreas de atuação condizentes com cada trabalho prático disponível na
plataforma.

Para além disso, torna-se imperativo criar um modelo genérico de publicação de
trabalhos práticos para dessa forma se tornar muito mais acessível o consumo de
informação para os utilizadores da plataforma.

No entanto também se torna necessário o foco no lado mais académico e das
necessidades dos docentes e alunos. Neste sistema todo o processo de
interatividade entre os utilizadores e a aplicação, permite tanto aos docentes
como aos alunos realizarem a gestão das suas unidades curriculares e trabalhos
práticos criando assim um sistema que cumpre as necessidades académicas internas dos docentes
e alunos bem como estabelece um ponto de ligação entre a comunidade académica e
o trabalho desenvolvido nesse meio, e o resto das pessoas com interesse nas
áreas abrangidas pelos projetos disponíveis.

\newpage

\section{Análise de requisitos}

Identificado o problema, pretende-se assim desenvolver um sistema de informação para gestão de trabalhos práticos. O sistema oferecerá um conjunto de funcionalidades e facilidades aos docentes e alunos em todo o processo de entrega de um trabalho prático, desde a gestão da unidade curricular até à publicação das avaliações.\\

Identifica-se à partida 3 grupos de utilizadores com funcionalidades e objetivos diferentes em todo o processo.\\

O primeiro grupo de utilizadores são os \textbf{alunos}, para estes pretende-se oferecer a possibilidade de submeterem projetos práticos de uma forma simples e facilitada. As submissões poderão ser efetuadas individualmente ou em grupos conforme o especificado pelo docente. O aluno terá acesso a um painel de gestão das suas unidades curriculares e projetos, poderá consultar todas as informações de um projeto e respetivas fases, gerir facilmente os seus grupos de trabalho, consultar o seu histórico de entregas e consultar em diferentes formatos as suas avaliações dos projetos e/ou fases. Pretende-se ainda que o aluno tenha a possibilidade de ser notificado de todas as alterações nos projetos que faz parte, através de avisos do sistema ou emails. Estas notificações serão também importantes para informar o aluno da aproximação dos prazos de entrega dos projetos das suas unidades curriculares.\\
No processo de submissão de um projeto, o sistema, deverá ser capaz de facilitar a criação do \textit{Project Record} do pacote enviado, assim como identificar e notificar falhas na submissão, tais como não conter todos os ficheiros obrigatórios ou não gerar o executável pretendido. Neste caso os alunos poderão receber no email essa informação para assim submeterem uma versão corrigida.
Um aluno terá também a possibilidade de tornar o projeto desenvolvido público.\\

O segundo grupo de utilizadores são os \textbf{docentes}, numa primeira componente estes utilizadores poderão criar e gerir unidades curriculares e todas as informações adjacentes a estas, um docente de uma unidade curricular pode adicionar docentes responsáveis, adicionar alunos e associar-los a diferentes turnos. Dentro da gestão de uma unidade curricular um docente será capaz de criar projetos devidamente documentados (organizados ou não em diferentes fases), podendo ainda especificar ficheiros obrigatórios, executável obrigatório e ficheiros de teste para facilitar a correção dos mesmos. Estes testes serão executados no programa enviado e serão guardadas as diferenças do output obtido para o output esperado. Dentro do painel de gestão de um projeto o docente poderá consultar as entregas dos diferentes grupos e avaliar as entregas por grupo ou individualmente. Depois de avaliadas as entregas, o docente, poderá gerar automaticamente as pautas do projeto e/ou fase e publicar para os alunos da unidade curricular.
Numa fase final, o docente poderá tornar o projeto (enunciado e especificações do problema) público.\\

Para além dos docentes e dos alunos, o nosso sistema permitirá uma procura e consulta de projetos públicos desenvolvidos a \textbf{utilizadores não registados}.\\

Pretende-se no desenvolvimento do sistema simplificar todas as tarefas dos utilizadores alvo do sistema, e proporcionar um sistema flexível que possa ser utilizado por um abrangente número de utilizadores.\\

Nesse seguimento o sistema procurará ser flexível, não se restringindo a apenas uma abordagem nem sendo demasiado genérico, procurar-se-à definir abordagens mais especificas, mas sempre com opções mais genéricas para englobar casos menos comuns. É exemplo disso a geração automática do \textit{Project Record} do pacote enviado, em casos mais específicos o ficheiro poderá ser entregue pelo utilizador, no entanto em outros casos torna-se necessário auxiliar na geração desse mesmo ficheiro para assegurar que todos os pacotes enviados estão em conformidade com a estrutura esperada de um pacote. O sistema também procurará simplificar processos organizacionais, como facilitar a criação de uma unidade curricular identificando apenas nome, instituição, curso e ano letivo. O facto de uma unidade curricular estar associada a um ano letivo ajuda a nível organizacional e permite que não seja necessária a transferência de alunos e docentes entre anos letivos. Procurará também simplificar processos de validação das entregas através das restrições de ficheiros obrigatórios e nome do executável. Ser capaz de automatizar a avaliação através dos testes no sistema e geração de pautas automáticas e de gerir mais facilmente grupos de trabalho e respetivas entregas.\\

O objetivo final é lançar um sistema bem focado onde cada grupo de utilizadores execute com facilidade as suas tarefas principais, perdendo o menor tempo possível em problemas secundários. 

\section{Concorrência e alternativas}

Uma das medidas efectudas durante o planeamento da aplicação foi a análise de concorrência e o estudo das alternativas existentes. Desta forma foi possivel adicionar mais funcionalidades à aplicação de forma a que esta fosse o mais completa possível.

\subsection{Concorrência} % (fold)
\label{sub:concorrencia}

De todas as aplicações concorrentes, as que mais se destacam são os Sistemas de Gestão de Aprendizagem (\emph{SGA}),também conhecidas por \emph{Learning Management Systems} - \emph{LMS}. Dentro dos \emph{SGA} imperam as seguintes aplicações:
\begin{itemize}
	\item \emph{Blackboard}
	\item \emph{Moodle}
	\item \emph{Studifi}
	\item \emph{Haiku LMS}
\end{itemize}

\subsubsection{Blackboard} % (fold)
\label{ssub:blackboard}

\begin{figure}[H]
        \centering
        \includegraphics[width=0.5\textwidth]{images/concorrencia/blackboard.jpg}
         \caption{\emph{Blackboard}}
         \label{fig: blackboard}
\end{figure}

A aplicação \href{http://www.blackboard.com}{\emph{Blackboard}} é composta por quatro tipos de utilizadores:
\begin{description}
	\item[Aluno] Um aluno pode submeter projetos, inscrever-se em grupos e consultar a informação disponibilizada pelos professores.
	\item[Professores] Um professor pode criar projetos e fazer avaliações qualitativas e quantitativas dos trabalhos enviados.
	\item[Técnicos] Os técnicos são responsáveis pela manutenção da aplicação.
	\item[Supervisores] Os supervisores fazem a gestão de cursos,disciplinas,docentes e alunos. Estes também podem gerar estatísticas do sistema.
\end{description}

Os projetos criados na \emph{Blackboard} não estão disponíveis para o público. Também não existe a noção de turnos, sendo estes substituidos pelos grupos. Os grupos são criados dentro da disciplina e um aluno pode estar inscrito em múltiplos grupos.

\subsubsection{Moodle} % (fold)
\label{ssub:moodle}
\begin{figure}[H]
        \centering
        \includegraphics[width=0.5\textwidth]{images/concorrencia/moodle.jpg}
         \caption{\emph{Moodle}}
         \label{fig: moodle}
\end{figure}
A aplicação \href{http://www.moodle.org}{\emph{Moodle}} é composta por três tipos de utilizadores:

\begin{description}
	\item[Alunos] Os alunos podem submeter projetos e consultar a informação disponibilizada pelos professores.
	\item[Professores] Os professores podem criar projetos e fazer uma avaliação quantitativa dos projetos enviados.
	\item[Administradors] Os Admnistradores são responsáveis pela manutenção do aplicação. Também são responsáveis por fazer a gestão dos utilizadores existentes.
\end{description}

Um dos problemas do \emph{Moodle} é a inexistência de grupos e de turnos. Quanto a projetos públicos apenas se estes forem inseridos manualmente pelos alunos nos seus portefólios.

\subsubsection{Studifi} % (fold)
\label{ssub:studifi}

\begin{figure}[H]
        \centering
        \includegraphics[width=0.5\textwidth]{images/concorrencia/studifi.png}
         \caption{\emph{Studifi}}
         \label{fig: studifi}
\end{figure}

A aplicação \href{https://studifi.com/}{\emph{Studifi}} é composta por três tipos de utilizadores:

\begin{description}
	\item[]
\end{description}


\subsection{Alternativas} % (fold)
\label{sub:alternativas}

% subsection alternativas (end)

\section{Casos de estudo}


No primeiro caso pretende-se que haja uma plataforma para a submissão do Projeto Integrado de Engenharia de Linguagens do Mestrado em Engenharia Informática.\\
Engenharia de Linguagens tem cerca de 25 alunos, 1 único turno e 4 docentes.
Quanto ao Projeto Integrado, este tem que ser feito por grupos com o máximo de 3 elementos, existem 4 fazes de entrega(as 3 primeiras tem uma nota qualitativa e a última fase vale 20 valores). Relativamente à entrega, em cada fase é obrigatório entregar um

\section{Arquitetura do sistema}
\subsection{Modelo OAIS}
O nosso sistema será construido seguindo as orientações do modelo OAIS \textit{(Open Archival Information System)} da Figura ~\ref{fig:oais}

\begin{figure}[H] 
  \centering
  \includegraphics[width=1\textwidth,center]{images/arquitetura/oais}
  \caption{Modelo OAIS}
  \label{fig:oais}
\end{figure}

Como representado na Figura ~\ref{fig:oais}, o sistema irá interagir com três tipos distintos de atores:
\begin{description}[labelindent=1cm]
  \item[Produtores] que serão representados pelos Alunos.
  \item[Administrador] que serão representados pelos Docentes.
  \item[Consumidor] que serão representados por todos os utilizadores do sistema, registados ou não.
\end{description}
E será constituido por três mega processos:
\begin{description}[labelindent=1cm]
  \item[Ingestão] responsável pela receção e depósito de projetos.
  \item[Administração] responsável pela gestão interna do sistema.
  \item[Disseminação] responsável pela disseminação, distribuição e publicação dos objetos arquivados.
\end{description}

\subsection{Fluxo do Sistema}

Com o objetivo de perceber e organizar melhor o fluxo do nosso sistema, construimos um \textbf{Diagrama de Atividade} que de uma forma pouco pormenorizada demonstra sequencialmente as principais ações dos utilizadores do sistema. De notar que este diagrama será importante para perceber a ligação entre os vários utilizadores e as ações dependentes entre estes.

Com o \textbf{Diagrama de Atividade} da Figura ~\ref{fig:diagrama-blocos}, podemos consultar de forma simplificada o fluxo do nosso sistema.

\begin{figure}[H] 
  \centering
  \includegraphics[width=1\textwidth,center]{images/arquitetura/diagrama-blocos}
  \caption{Fluxo do Sistema}
  \label{fig:diagrama-blocos}
\end{figure}

Para a representação do fluxo das funcionalidades mais complexas da aplicação, construiu-se \textbf{Diagramas de Atividade} que permitem representar essas mesmas funcionalidades mais pormenorizadamente.
Estes diagramas para além de ajudarem a perceber melhor o funcionamento do sistema nas tarefas mais relevantes, serão um excelente apoio na implementação do sistema.

Na Figura ~\ref{fig:criacao-projecto} podemos consultar o fluxo de criação de um projeto, desde o acesso ao sistema por parte de um docente, até ao acesso à página do projeto por parte de um aluno.

\begin{figure}[H] 
  \centering
  \includegraphics[width=1\textwidth,center]{images/arquitetura/criacao-projecto}
  \caption{Fluxo: Criar Projecto}
  \label{fig:criacao-projecto}
\end{figure}

Na Figura ~\ref{fig:submissao-projecto} podemos consultar o fluxo de submissão de um projeto, desde o acesso à página de um projeto por parte de um utilizador até à consulta das informações da entrega por parte de um docente.

\begin{figure}[H] 
  \centering
  \includegraphics[width=1\textwidth,center]{images/arquitetura/submissao-projecto}
  \caption{Fluxo: Submeter Projeto}
  \label{fig:submissao-projecto}
\end{figure}

\section{Modelo de dados}

Uma das componentes importantes do desenho de um sistema de informação, passa por criar um modelo que explique as características de funcionamento e comportamento do \textit{software} a ser desenvolvido. A modulação para além de ajudar na compreensão do sistema, evita erros de programação, de projeto e de funcionamento.

\subsection{Modelo de Domínio}

Um dos primeiros passos na planificação do nosso sistema passou pela definição do \textbf{Modelo de Domínio}.
Podemos classificar como domínio do nosso sistema, o conjunto de características que descrevem a família de problemas que a nossa aplicação pretende solucionar.

Podemos consultar na Figura ~\ref{fig:modelo-dominio} os termos do nosso sistema e as relações existentes entre esses termos.

\begin{figure}[H] 
  \centering
  \includegraphics[width=1\textwidth,center]{images/modelo_dados/modelo-dominio}
  \caption{Modelo de domínio}
  \label{fig:modelo-dominio}
\end{figure}

\section{Repositório de informação}

Para a representação do repositório de informação do sistema, utilizar-se-á um \textbf{Diagrama entidade relacionamento}, que descreve o modelo de dados de um sistema com alto nível de abstração. 

Na Figura ~\ref{fig:der} pode-se consultar todas as tabelas que constituirão o sistema, assim como os seus atributos. De notar que os tipos especificados para os atributos podem não retratar corretamente os tipos que serão utilizados. Como se irá utilizar a \textit{framework} Ruby on Rails para desenvolvimento, esta permite uma maior abstração em relação aos tipos dos atributos da base de dados. Assim sendo os tipos serão definidos segundo o Diagrama de Classes apresentado anteriormente e a \textit{framework} fará a conversão automática conforme a base de dados que estivermos a utilizar.

\begin{figure}[H] 
  \centering
  \includegraphics[width=1\textwidth,center]{images/repositorio_informacao/der}
  \caption{Diagrama Entidade Relacionamento}
  \label{fig:der}
\end{figure}

De notar que será utilizado \textbf{SQLite} em ambientes de desenvolvimento e testes, e \textbf{PostgreSQL} para ambientes de produção.

\section{Funcionalidades do sistema}

No que diz respeito às funcionalidades do sistema é necessário ter em conta a definição dos seus 
utilizadores e quais as objetivos que a aplicação se propõe a resolver.
Nesse sentido foram definidos três tipos de utilizadores cada um com um conjunto concreto de 
funcionalidades que lhes são fornecidas.
De forma a conseguir fazer este tipo de especificação foram idealizados os seguinte tipos de utilizadores
que interagirão com o sistema, sendo eles: utilizador não registado e utilizador registado na forma de 
docente e aluno.

\begin{figure}[H] 
   \centering
   \includegraphics[width=1\textwidth]{images/funcionalidades/usecases.png}
    \caption{Diagrama com casos de uso dos utilizadores do sistema}
    \label{fig: usecases}
 \end{figure}

\subsection{Utilizadores não registados}

Como é possível ver no diagrama de Casos de uso representado na Figura \ref{fig: usecases}, no caso 
dos Utilizadores não registados, representados no diagrama pelo ator Visitante, são disponibilizadas
funcionalidades de registo, entrada no sistema, consulta e \textit{download} de projetos públicos.

Assim sendo é permitido a este tipo de utilizadores registarem-se na aplicação, providenciando informações
pessoais e criando então uma conta persistente com os seus dados, definindo-se a si próprio como
docente ou aluno. Após o registo é-lhes permitida a entrada no sistema com as credenciais de acesso
definidas aquando do registo prévio. Como utilizadores não registados podem também consultar e procurar
todos os projetos disponibilizados por outros utilizadores na aplicação, assim como também lhes é fornecida
a opção de descarregarem os mesmos projetos para o dispositivo de onde acessam.

\subsection{Docentes}

O seguinte tipo de utilizadores descrevem-se como utilizadores registados, sendo-lhes 
disponibilizadas funcionalidades específicas dentro do ambiente da aplicação. Nesse sentido, para 
além de serem uma extensão de Utilizadores não registados após o registo, são-lhes conferidas um
grupo de funcionalidades adicionais que descrevem o comportamento deste tipo de utilizador.

A um docente são disponibilizados quatro grupos de funcionalidades, como é possível observar na Figura
\ref{fig: usecases}. Esses grupos de funcionalidades são a gestão de unidades curriculares, turnos, 
projetos e fases.

No que diz respeito à gestão de unidades curriculares é tornado possível a este tipo 
de utilizador vários tipos de ações sobre as mesmas, nomeadamente, a consulta, a criação e 
remoção, a gestão de docentes, atualização das informações associadas, entre 
outras.

Para além disso, dentro de uma unidade curricular também é possível proceder-se 
à gestão de turnos. Nesse aspeto, é providenciado ao docentes tarefas como consulta, 
criação e remoção de turnos bem como a adição e remoção de alunos desses mesmo 
turnos.

Relativamente à gestão dos projetos de uma unidade curricular são fornecidas 
aos docentes as ferramentas para a consulta, criação e remoção, publicação, 
estabelecimentos de prazos assim como limitações ao número de elementos de um 
grupo, avaliação de um elemento ou grupo e por fim para geração de pautas, de 
todos os projetos associados a uma unidade curricular onde o docente seja 
responsável.

Por fim, a gestão de fases de um projeto é definida por funcionalidades como a 
sua criação e remoção, consulta, tipo de visibilidade, definição de 
obrigatoriedade de ficheiros, definição de testes sobre entregas e estabelecimento de prazos limite 
para submissões.

\subsection{Alunos}

Tal como o tipo de utilizador Docente, o tipo de utilizador Aluno é uma extensão 
de um utilizador não registado após o registo. Este tipo de utilizador é 
definido pelas funcionalidades que lhe são disponibilizadas pelo sistema. Dentro 
destas funcionalidades podemos destacar um conjunto delas, nomeadamente a 
consulta de projetos e unidades curriculares, a gestão de grupos e de submissões de projetos.

No que diz respeito à gestão de grupos é possível um aluno criar um grupo dentro 
de um projeto e adicionar e remover outros alunos como elementos do grupo.

Para além disso também são disponibilizadas ferramentas na aplicação que permite 
aos alunos fazerem submissões ou entregas de projetos dentro de uma unidade 
curricular de onde façam parte de um dos turnos. Para além disso também lhes é 
possível consultar as avaliações dadas pelos docentes às suas submissões assim 
como consultar a unidade curricular e as pautas da mesma.

\newpage

\section{Protótipos para a interface da aplicação}

Um dos processos de desenvolvimento de uma Aplicação Web é a criação de protótipos. Este processo tem como objetivo demonstrar de que forma o utilizador interage com a aplicação e ajuda a tomar decisões antes da implementação da aplicação. Desta forma há um maior entendimento entre o utilizador e a equipa de desenvolvimento e aumenta a usabilidade da aplicação por parte do utilizador, uma vez que são garantidas as seguintes habilidades:

\begin{description}
	\item[Fácil apendizagem] A utilização da aplicação requer pouco treino
	\item[Fácil de memorizar] O utilizador lembra-se de como utilizar a interface depois algum tempo
	\item[Maximizar a produtividade] A realização de uma tarefa é feita de forma rápida e eficiente
	\item[Minimizar a taxa de erros] A aplicação avisa o utlizador, caso existam erros e ajuda na correção dos mesmos
	\item[Maximizar a satisfação] A aplicação transmite confiança e segurança ao utilizador
\end{description}

\subsection{Primeiros protótipos}

Um dos processos de desenvolvimento de uma Aplicação Web é a criação de protótipos (\emph{Mockups}) que demonstrem de como será o aspeto visual da nossa aplicação.\\
Durante a produção dos primeiros protótipos, teve-se como objetivo identificar de que formas as várias funcionalidades do sistemas iriam estar presentes na aplicação, sem haver preocupações com questões estéticas.Este protótipos foram desenvolvidos usanto a aplicação \href{http://balsamiq.com/products/mockups/}{\emph{Balsamiq Mockups}}(Figura ~\ref{fig: balsamiq}).\\

\begin{figure}[htbp] 
        \centering
        \includegraphics[width=1\textwidth]{images/prototipos/mockups/balsamiq.png}
         \caption{\emph{Balsamiq Mockups}}
         \label{fig: balsamiq}
\end{figure}


%Introdução a falar dos mockups feitos
Sobre os prótipos feitos, teve-se como preocupação a simplicidade da pagina, a quantidade de informação e o tempo de navegação, isto é a quantidade de ações que um utilizador tem que fazer para passar de uma página para outra.\\
Começando pela página inicial (Figura ~\ref{fig: home}) está mostra as várias funcionalidades da aplicação, fazendo a distinção entre alunos e docentes.Também é possivel aceder aos projetos públicos do sistema.\\

\begin{figure}[htbp] 
        \centering
        \includegraphics[width=1\textwidth]{images/prototipos/mockups/home.png}
         \caption{Página inicial}
         \label{fig: home}
\end{figure}

No painel de uma disciplina(Figura ~\ref{fig: cursodocente}), um docente pode ter acesso aos últimos acontecimentos dentro das disciplina, sabendo quando aconteceram submissões nos projetos desta, assim alterações da mesma.Também existem informações sobre os projetos criados na disciplina, no qual são apresentadas informações sobre o nome do prjeto, número de fases,estado e número de entregas até ao momento. No caso de ser o docente responsável da disciplina podes adicionar professores de forma direta sem que haja a necessidade de abrir formulários adicionais e fazer gestão de turnos. Caso contrário apenas poderá ver o estado dos turnos e quais são os docentes da disciplina.\\

\begin{figure}[htbp] 
        \centering
        \includegraphics[width=1\textwidth]{images/prototipos/mockups/cursodocente.png}
         \caption{Painel de projeto de um aluno}
         \label{fig: cursodocente}
\end{figure}

Na paindel de pesquisa de projetos(Figura ~\ref{fig: projetospublicos}), todos os utilizadores tem acesso aos projetos públicos. Nesta página o utilizador pode filtrar uma pesquisa de forma a que o processo de pesquisa seja mais rapido. Os resultados de uma pesquisa são apresentados sob forma de blocos, desta forma, garante-se um maior aproveitamento do espaço livre da pagina.\\ 

\begin{figure}[htbp] 
        \centering
        \includegraphics[width=1\textwidth]{images/prototipos/mockups/Projetos.png}
         \caption{Listagem dos projetos públicos}
         \label{fig: projetospublicos}
\end{figure}


Quando dentro de uma 

\begin{figure}[htbp] 
        \centering
        \includegraphics[width=1\textwidth]{images/prototipos/mockups/painelprojetodocente.png}
         \caption{Painel de projeto de um docente}
         \label{fig: painelprojetodocente}
\end{figure}

\begin{figure}[htbp] 
        \centering
        \includegraphics[width=1\textwidth]{images/prototipos/mockups/painelprojetoaluno.png}
         \caption{Painel de projeto de um aluno}
         \label{fig: painelprojetoaluno}
\end{figure}




\begin{figure}[htbp] 
        \centering
        \includegraphics[width=1\textwidth]{images/prototipos/mockups/projetodocente.png}
         \caption{Entrega vista por um docente}
         \label{fig: projetodocente}
\end{figure}


\begin{figure}[htbp] 
        \centering
        \includegraphics[width=1\textwidth]{images/prototipos/mockups/projetovisitante.png}
         \caption{Entrega vista por um visitante}
         \label{fig: projetoaluno}
\end{figure}










\subsection{Protótipos fidedignos}
Para além dos primeiros protótipos, fizeram-se também protótipos fidedignos baseados nos anteriormente
citados de forma a aumentar o nível de detalhe, bem como aproximar a prototipagem o mais 
próximo possível do resultado final esperado. Assim sendo, segue-se na figura \ref{fig:prot_fid_home}
um exemplo da prototipagem fidedigna da página principal.
Com este tipo de protótipos espera-se aprimorar os detalhes, bem como os esquemas de cores,
arranjos gráficos, entre outros pormenores.
De notar que os protótipos fidedignos feitos neste projeto foram construídos diretamente a partir
da linguagem HTML.

\begin{figure}[H] 
  \centering
  \includegraphics[width=0.8\textwidth,center]{images/prototipos/fidedigno_home.jpg}
  \caption{Protótipo fidedigno página inicial}
  \label{fig:prot_fid_home}
\end{figure}

\newpage


\section{Proposta tecnológica}

Para o desenvolvimento do nosso sistema, precisamos de combinar um vasto conjunto de tecnologias que permitam 
simplificar o desenvolvimento de todas as funcionalidades propostas.

De forma a organizar o fluxo de trabalho do grupo, utilizamos o \href{https://trello.com/}{\textbf{Trello}}, 
que é um sistema para gestão de tarefas que segue o método \textit{kanban}.
\begin{figure}[H] 
  \centering
  \includegraphics[width=1\textwidth]{images/tecnologias/trello}
  \caption{Trello}
  \label{fig:trello}
\end{figure}

Para o controlo de versões utilizaremos \textbf{Git}, que é um sistema de controlo de versões 
centralizado que apresenta as seguintes características:

\begin{itemize}
  \item Suporte consistente para desenvolvimentos não-lineares
  \item Desenvolvimento distribuído
  \item Compatibilidade com protocolos/sistemas existentes
  \item Manipulação eficiente de projetos extensos  
  \item Autenticação criptográfica do histórico 
  \item Modelo baseado em ferramentas 
  \item Estratégias de mescla (merge) conectáveis
  \item Empacotamento periódico explícito de objetos  
\end{itemize}

Para armazenar o código do nosso projeto vamos utilizar o \href{http://github.com}{\textbf{Github}}, 
que é um serviço de \textit{web hosting} o para projetos que usam o \textbf{Git} como sistema de controle de versões. 
Iremos utilizar um sistema de \textit{\textbf{Pull Requests}} e \textit{\textbf{Code Reviews}} 
para que assim todos os elementos possam ver e discutir o código produzido.

Procuraremos respeitar um dos fluxos de trabalho baseados em Git mais conhecidos, 
o \href{https://www.atlassian.com/git/workflows#!workflow-feature-branch}{\textbf{\textit{Feature Branch Workflow}}} 
que pode ser representado pela figura ~\ref{fig:git-workflow}.

\begin{figure}[H] 
  \centering
  \includegraphics[width=0.5\textwidth]{images/tecnologias/git-workflow}
  \caption{Feature Branch Workflow}
  \label{fig:git-workflow}
\end{figure}

Relativamente ao desenvolvimento, utilizaremos \href{http://rubyonrails.org/}{\textbf{Ruby on Rails}}, 
que é uma \textit{framework open-source} escrita em \textbf{Ruby}, otimizada para a produtividade sustentável. 
A \textit{framework} tem uma forte comunidade que defende conceitos como \textbf{DRY} \textit{(Don't Repeat Yourself)} 
e \textbf{\textit{Convention over configuration}}.

\textbf{Ruby on Rails} é uma \textit{meta-framework}, composta pelos seguintes \textbf{frameworks}:

\begin{description}[labelindent=1cm]
  \item[Active Record] Responsável pela interoperabilidade entre a aplicação e a base de dados, e pela abstração dos dados.
  \item[Action Pack] Compreende o \textbf{Action View} (gera o que o utilizador vê, como HTML, XML e JavaScript) 
  e o \textbf{Action Controller} (controle do fluxo de negócio).
  \item[Action Mailer] Responsável pelo serviço de entrega e receção de e-mails.
  \item[Active Support] Coleção de várias classes e bibliotecas, que foram considerados úteis para aplicações em Ruby on Rails.
\end{description}

Para além das \textit{frameworks} indicadas iremos recorrer a \textbf{Gems} para simplificar alguns processos, tais como:

\begin{description}[labelindent=1cm]
  \item[Devise] Autenticação de utilizadores.
  \item[Paperclip] Upload e validação de imagens e arquivos.
  \item[Capistrano] Automatização de processos de implantação \textit{(deployment)}.
  \item[Nokogiri] Parser de HTML e XML com XPath e seletores CSS.
  \item[XML-Simple] API simples para trabalhar com documentos XML.
\end{description} 

Pretendemos melhorar a nossa produtividade, e melhorar a legibilidade do código produzido utilizando os seguintes pré-processadores:

\begin{description}[labelindent=1cm]
  \item[\href{http://slim-lang.com/}{Slim}] para HTML.
  \item[\href{http://sass-lang.com/}{SASS}] para CSS.
  \item[\href{http://coffeescript.org/}{CoffeeScript}] para JavaScript.
\end{description} 

Utilizaremos a \textit{framework} de \textit{front-end} \href{http://getbootstrap.com/}{\textbf{Twitter Bootstrap}} 
para ajudar a construirmos um serviço multi-plataforma e para reutilizarmos alguns componentes.
Para além disso também utilizaremos a biblioteca de Javascript \href{http://jquery.com/}{\textbf{JQuery}} para simplificar os \textit{scripts client side} que interagem com o HTML.\\

Para manipulação de documentos \textbf{XML}, utilizaremos as tecnologias lecionadas, tais como: \textbf{DTD}, \textbf{XSLT} e \textbf{XSD}.\\

Relativamente a tecnologias de Base de Dados, utilizaremos bases de dados SQL, mais especificamente \href{http://www.sqlite.org/}{\textbf{SQLite}} 
para ambientes de teste e desenvolvimento, e \href{http://www.postgresql.org/}{\textbf{PostgreSQL}} para ambientes de produção.

\newpage

\section{Estado do desenvolvimento}
No sentido de colocar em prática todo o planeamento do desenvolvimento da aplicação, seguiu-se de forma minimamente próxima o plano de trabalho já demonstrado na forma de diagramas de Gantt com pequenas alterações de prioridade no que diz respeito à implementação de algumas fases.

Nesse sentido após a construção da base da aplicação nas tecnologias que seriam usadas procedeu-se à prototipagem em HTML de algumas páginas.

Neste momento é possível encontrar já em fase quase final a página principal da aplicação que sofreu algumas iterações até se aproximar do pretendido.

Passado esta primeira fase de prototipagem começou-se por garantir a persistência de dados através da implementação do modelo de dados concebido durante a análise de requisitos bem como proceder à sua população com dados de teste.

No que diz respeito aos diferentes utilizadores das aplicação e à sua autenticação no sistema foram implementadas funcionalidades de registo e entrada  no sistema de forma a garantir uma gestão fiel das funcionalidades que cada tipo de utilizador tem acesso, bem como garantir coerência de dados para cada utilizador individual.

Após garantir a autenticação no sistema aos seus utilizadores,bem como o registo específico para cada um dos tipos de utilizadores (docentes e alunos), iniciou-se aimplementação das funcionalidades básicas do sistema, nomeadamente o \textit{CRUD (create, read, updateand delete)} da aplicação, ou seja, a criação, leitura, actualização e remoção de todas as entidades representadas no sistema (como por exemplo turnos, projetos, grupos, avaliações, etc).

Após uma implementação básica dessas funcionalidades decidiu-se iterar sobre as funcionalidades ligadas ao tipo de utilizador aluno e refazê-las de raiz de forma a ir de encontro ao que realmente são as funcionalidades que pretendemos disponibilizar aos alunos da aplicação.

Nesse sentido iniciou-se o desenvolvimento do painel do aluno que se torna o  centro de utilização da aplicação por parte deste tipo de utilizador.
Aqui é possível consultar os projetos em desenvolvimento por parte do utilizador, assim como os projetos que se vão iniciar e os que já foram terminados.
Também é possível consultar as últimas notificações do sistema assim como se tornam acessíveis todas as unidades curriculares em que cada aluno se encontra inscrito.

De seguida prosseguiu-se ao aprofundamento da implementação das funcionalidades sobre projetos do lado do aluno, tendo sido desenvolvida uma página onde é possível consultar as informações disponibilizadas para o projeto, bem como as fases, as suas notificações, o grupo de trabalho e as últimas entregas.
A partir daqui é também possível consultar páginas de entrega e fases de um projeto onde são disponibilizadas informações como os ficheiros obrigatórios, a pauta de avaliações, as entregas feitas e onde é possível efectuar novas entregas.
Para além disso também se tornam acessíveis na página de um projeto páginas que permitem a gestão de grupos e consulta de enunciados.

Ainda do lado do aluno e no que diz respeito às unidades curriculares em que estão inscritos é disponibilizada um página que permite a procura de unidades curriculares e a sua inscrição mediante a aceitação por parte do docente do respetivo pedido, bem como a funcionalidade de consulta das unidades curriculares em que está inscrito e as que estão com a inscrição pendente, assim com a opção para anular uma inscrição.
Assim, é também tornada possível a consulta de todas as avaliações das diferentes unidades curriculares em que um aluno se encontra inscrito.

De referir que todas as funcionalidades que já foram implementadas já se encontram numa fase de maturação considerável e que estão bastante próximas da sua versão final apesar de estarem ainda sujeitas a pequenas alterações, tanto no sentido do seu funcionamento como no sentido da própria interface, que por sua vez já se encontra preparada para o acesso a partir de várioss tipos de dispositivos independentemente da resolução.

Em jeito de resumo já é possível fornecer aos alunos utilizadores do sistema uma aplicação que lhes permite gerir as suas unidades curriculares, ingressar em projetos, fazer entregas, consultar pautas e avalições, criar os seus grupos de trabalho, entre outras funcionalidades, garantido assim grande parte das funcionalidades planeadas para este tipo de utilizador.

À data de escrita deste relatório, já se podem dar como garantidas praticamente todas as funcionalidades não só dos utilizadores alunos como também dos utilizadores docentes. Também do lado dos utilizadores não registados é possível fazer pesquisa de projetos publicados na plataforma.

De realçar também a implementação de uma API 'restful` e de um utilitário de consola para servir de meio à interação com a aplicação. Neste momento encontram-se implementados métodos básicos no que diz respeito aos pedidos que um utilizador docente pode fazer à aplicação. No futuro tratar-se-ão das funcionalidades mais avançadas e da interação dos utilizadores alunos através da mesma API.

\newpage

\section{Conclusão}
Chegado a fase final do projecto torna-se possível e necessário destacar algumas conclusões e ilações que merecem destaque durante toda esta última fase de desenvolvimento e término do projecto.

Esta última fase mostrou ser uma fase de enriquecimento e polimento do projecto tentando sempre estender o seu alcance a todos os conteúdos leccionados durante todo o ano lectivo na UCE de Engenharia de Linguagens.

Neste sentido, nesta última fase tentou-se chegar a um ponto de convergência entre todos os módulos sendo possível usar todo o tipo de tecnologias que abordámos. É possível destacar o enriquecimento da API restful já iniciada na etapa anterior, assim como o uso de ferramentas de processamento de texto como o Apace FOP para XSLFO. Para além disso também se deve destacar o desenvolvimento de um utilitário para a consola desenvolvido em Perl onde é possível fazer pedidos e interagir com a aplicação sem ter que se optar pela interface web.

É também importante focar que uma aplicação desta complexidade tem sempre um espaço de evolução enorme, e como tal seria possível acrescentar todo um rol de funcionalidades novas e expandir o sistema, visto que ele se encontra preparado para tal.

A aplicação web foi dada como terminada, bem como todas as tarefas e objectivos aos quais, ambiciosamente nos tínhamos propostos. Assim sendo encerramos o processo de desenvolvimentos satisfeitos com o resultado final e acima de tudo com todo o processo de aprendizagem como grupo e como alunos da UCE de Engenharia de Linguagens.

\newpage

\section{Anexos}
\begin{figure}[H] 
	\centering
	\includegraphics[width=0.3\textwidth]{images/plano_de_trabalho/gannt_0.png}
 	\caption{Diagrama de Gannt completo}
 	\label{fig: workplan0}
\end{figure}

\input{bibliografia}

\end{document}

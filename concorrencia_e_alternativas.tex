\section{Concorrência e alternativas}

Uma das medidas efectudas durante o planeamento da aplicação foi a análise de concorrência e o estudo das alternativas existentes. Desta forma foi possivel adicionar mais funcionalidades à aplicação de forma a que esta fosse o mais completa possível.

\subsection{Concorrência} % (fold)
\label{sub:concorrencia}

De todas as aplicações concorrentes, as que mais se destacam são os Sistemas de Gestão de Aprendizagem (\emph{SGA}),também conhecidas por \emph{Learning Management Systems} - \emph{LMS}. Dentro dos \emph{SGA} imperam as seguintes aplicações:
\begin{itemize}
	\item \emph{Blackboard}
	\item \emph{Moodle}
	\item \emph{Studifi}
	\item \emph{Haiku LMS}
\end{itemize}

\subsubsection{Blackboard} % (fold)
\label{ssub:blackboard}

\begin{figure}[H]
        \centering
        \includegraphics[width=0.5\textwidth]{images/concorrencia/blackboard.jpg}
         \caption{\emph{Blackboard}}
         \label{fig: blackboard}
\end{figure}

A aplicação \href{http://www.blackboard.com}{\emph{Blackboard}} é composta por quatro tipos de utilizadores:
\begin{description}
	\item[Aluno] Um aluno pode submeter projetos, inscrever-se em grupos e consultar a informação disponibilizada pelos professores.
	\item[Professores] Um professor pode criar projetos e fazer avaliações qualitativas e quantitativas dos trabalhos enviados. Os projetos criados podem ter correção automática.
	\item[Técnicos] Os técnicos são responsáveis pela manutenção da aplicação.
	\item[Supervisores] Os supervisores fazem a gestão de cursos,disciplinas,docentes e alunos. Estes também podem gerar estatísticas do sistema.
\end{description}

Os projetos criados na \emph{Blackboard} não estão disponíveis para o público. Também não existe a noção de turnos, sendo estes substituidos pelos grupos. Os grupos são criados dentro da disciplina e um aluno pode estar inscrito em múltiplos grupos.

\subsubsection{Moodle} % (fold)
\label{ssub:moodle}
\begin{figure}[H]
        \centering
        \includegraphics[width=0.5\textwidth]{images/concorrencia/moodle.jpg}
         \caption{\emph{Moodle}}
         \label{fig: moodle}
\end{figure}
A aplicação \href{http://www.moodle.org}{\emph{Moodle}} é composta por três tipos de utilizadores:

\begin{description}
	\item[Alunos] Os alunos podem submeter projetos e consultar a informação disponibilizada pelos professores.
	\item[Professores] Os professores podem criar projetos e fazer uma avaliação quantitativa dos projetos enviados. Os projetos criados podem ter correção automática.
	\item[Administradors] Os Admnistradores são responsáveis pela manutenção do aplicação. Também são responsáveis por fazer a gestão dos utilizadores existentes.
\end{description}

Um dos problemas do \emph{Moodle} é a inexistência de grupos e de turnos. Quanto a projetos públicos apenas se estes forem inseridos manualmente pelos alunos nos seus portefólios.

\subsubsection{Studifi} % (fold)
\label{ssub:studifi}

\begin{figure}[H]
        \centering
        \includegraphics[width=0.5\textwidth]{images/concorrencia/studifi.png}
         \caption{\emph{Studifi}}
         \label{fig: studifi}
\end{figure}

A aplicação \href{https://studifi.com/}{\emph{Studifi}} é composta por três tipos de utilizadores:

\begin{description}
	\item[Alunos] Os alunos podem submeter projetos e consultar a informação disponibilizada pelos professores.
	\item[Professores] Os professores podem criar exercicios que são avaliados automáticamente e podem criar projetos.
	\item[Administradors] Os Admnistradores são responsáveis pela manutenção do aplicação. Também são responsáveis por fazer a gestão dos utilizadores existentes.
\end{description}

No \emph{Studifi} os exercicios não estão disponiveis para o público. Uma das caraterísticas que se destaca nesta aplicação é a existencia de controlo de versões nos ficheiros submetidos.


\subsubsection{Haiku LMS} % (fold)
\label{ssub:haiku_lms}

\begin{figure}[H]
        \centering
        \includegraphics[width=0.5\textwidth]{images/concorrencia/studifi.png}
         \caption{\emph{Haiku LMS}}
         \label{fig: studifi}
\end{figure}

A aplicação \href{http://www.haikulearning.com/}{\emph{Haiku LMS}} é composta por três tipos de utilizadores:

\begin{description}
	\item[Alunos] Os alunos podem submeter projetos e consultar a informação disponibilizada pelos professores.
	\item[Professores] Os professores podem criar projetos e exercicios e gerir pautas de avaliação.
	\item[Instituições de ensino] Responsáveis pela gestão de cursos e disciplinas da instituição.
\end{description}

\subsection{Alternativas} % (fold)
\label{sub:alternativas}

% subsection alternativas (end)

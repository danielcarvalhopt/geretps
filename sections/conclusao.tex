\section{Conclusão}
Chegado o final da terceira fase deste projeto é possível destacar algumas conclusões e decisões que tiveram destaque durante esta etapa de desenvolvimento.

Esta fase do projeto mostrou ser uma fase de maturação e consolidação do estado do projeto. Foram concluídas quase todas as tarefas relativas às funcionalidades e interfaces da aplicação web, mantendo-se à margem disso apenas pequenas funcionalidades e pormenores na interface, como é o caso da implementação de funcionalidades relativas a testes que estão ainda por ser finalizadas.

Nesta fase a aplicação já tem uma dimensão considerável e tornou-se bastante desafiante manter o ritmo de desenvolvimento que tinha sido mantido até então. Não obstante, os objetivos que tinham sido definidos à partida para esta fase foram cumpridos.

Umas das decisões tomadas foi a implementação de uma API 'restful` com um utilitário para consola onde é possível fazer pedidos e interagir com a aplicação sem ter que se optar pela interface web. 

De forma geral, os objetivos foram atingidos com sucesso, no que diz respeito aos docentes já é possível fazer toda a gestão de Unidades curriculares, projetos, fases, turnos, grupos de alunos, entregas de grupo, bem como avalições. De forma análoga, também as funcionalidades dos utilizadores alunos estão praticamente concluídas. É também já possível ao público em geral fazer pesquisa dos projetos e trabalhos contidos na aplicação.

Para a próxima fase existe um comprometimento em terminar o trabalho que está pela frente e melhorar significativamente a aplicação e as suas funcionalidades.

\newpage

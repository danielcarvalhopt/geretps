\section{Conclusão}
Chegado a fase final do projecto torna-se possível e necessário destacar algumas conclusões e ilações que merecem destaque durante toda esta última fase de desenvolvimento e término do projecto.

Esta última fase mostrou ser uma fase de enriquecimento e polimento do projecto tentando sempre estender o seu alcance a todos os conteúdos leccionados durante todo o ano lectivo na UCE de Engenharia de Linguagens.

Neste sentido, nesta última fase tentou-se chegar a um ponto de convergência entre todos os módulos sendo possível usar todo o tipo de tecnologias que abordámos. É possível destacar o enriquecimento da API restful já iniciada na etapa anterior, assim como o uso de ferramentas de processamento de texto como o Apace FOP para XSLFO. Para além disso também se deve destacar o desenvolvimento de um utilitário para a consola desenvolvido em Perl onde é possível fazer pedidos e interagir com a aplicação sem ter que se optar pela interface web.

É também importante focar que uma aplicação desta complexidade tem sempre um espaço de evolução enorme, e como tal seria possível acrescentar todo um rol de funcionalidades novas e expandir o sistema, visto que ele se encontra preparado para tal.

A aplicação web foi dada como terminada, bem como todas as tarefas e objectivos aos quais, ambiciosamente nos tínhamos propostos. Assim sendo encerramos o processo de desenvolvimentos satisfeitos com o resultado final e acima de tudo com todo o processo de aprendizagem como grupo e como alunos da UCE de Engenharia de Linguagens.

\newpage

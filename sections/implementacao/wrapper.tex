\subsection{Interface PERL para a API do sistema}

Para facilitar a utilização da API do sistema, foi desenvolvida uma interface em PERL que permite comunicar com o sistema de uma forma mais simples.

Pretende-se assim facilitar o desenvolvimento de ferramentas que comuniquem com o sistema.\\

A referida interface foi desenvolvida recorrendo a vários módulos PERL, que são descritos de seguida:

\begin{itemize}
  \item \verb%Net::GereTPs% \\ Módulo responsável por reencaminhar o acesso para a versão correta da API.
  \item \verb%NET::GereTPs::V1% \\ Módulo responsável por reencaminhar os pedidos para os módulos respetivos, e armazenar as informações de autenticação.
  \item \verb%NET::GereTPs::V1::Session% \\ Módulo responsável pelos pedidos de autenticação.
  \item \verb%NET::GereTPs::V1::Projects% \\ Módulo responsável pelos pedidos ao controlador de projetos do sistema.
  \item \verb%NET::GereTPs::V1::Phases% \\ Módulo responsável pelos pedidos ao controlador de fases do sistema.
  \item \verb%NET::GereTPs::V1::Deliveries% \\ Módulo responsável pelos pedidos ao controlador de entregas do sistema.
  \item \verb%NET::GereTPs::V1::Documents% \\ Módulo responsável pelos pedidos ao controlador de documentos do sistema.
  \item \verb%NET::GereTPs::V1::Groups% \\ Módulo responsável pelos pedidos ao controlador de grupos do sistema.
\end{itemize}

De seguida pode consultar-se um exemplo de utilização da interface desenvolvida:

\begin{verbatim}
use Net::GereTPs;

my $geretps = Net::GereTPs->new({email => $params{email}, password => $params{password}});
my $auth_token = $geretps->session->get_auth_token();
my $projects = $geretps->projects->all();
my $group_info = $geretps->groups->get_xml($group_id);
my $groups = $geretps->groups("phases",$phase_id)->all();
\end{verbatim}

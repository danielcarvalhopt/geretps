\subsection{Sistema de emails}

Além das notificações da aplicação, um utilizador também pode ser notificado através de \texir{email}.

As diferenças em relação as notificações, são que um utilizador só é notificado via \texit{email}, nos seguintes casos:

\begin{itemize}
	\item Submissão de uma entrega
	\item Lançamento de notas
	\item Inscrição numa unidade curricular
	\item Admissão num grupo
\end{itemize}

No caso da submissão de uma entrega, todos os elementos do grupo serão notificados com um \texit{email} com a descrição da submissão e com \texit{links} para os ficheiros submetidos.

Na figura ~\ref{fig:mail} é possível ver o conteúdo de um \texit{email} que foi recebido após a submissão de uma entrega.

\begin{figure}[H]
	\centering
	\includegraphics[width=0.65\textwidth,center]{images/implementacao/alunos/emails}
	\caption{Notificações}
	\label{fig:mail}
\end{figure}

\subsection{Sistema de emails}

Além das notificações da aplicação, um utilizador também pode ser notificado através de \texit{email}.

Uma das grandes vantagens dos emails em relação às notificações é que o utilizador não precisa de aceder à aplicação para saber dos eventos mais importantes.

Algumas das situações em que um sistema de \texit{email} se torna  vantajoso são:

\begin{itemize}
	\item Submissão de uma entrega
	\item Lançamento de notas
	\item Inscrição numa unidade curricular
	\item Admissão num grupo
	\item Novo projeto
	\item Aproximação do final de uma fase
\end{itemize}

No caso da submissão de uma entrega, todos os elementos do grupo serão notificados com um \texit{email} com a descrição da submissão e com \texit{links} para os ficheiros submetidos. Neste caso o email pode ser importante como comprovativo da entrega.

Na figura ~\ref{fig:mail} é possível ver o conteúdo de um \texit{email} que foi recebido após a submissão de uma entrega.

\begin{figure}[H]
	\centering
	\includegraphics[width=0.65\textwidth,center]{images/implementacao/alunos/emails}
	\caption{Notificações}
	\label{fig:mail}
\end{figure}
